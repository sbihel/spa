\documentclass{beamer}

\graphicspath{{img/}}

\usepackage{appendixnumberbeamer}

\usepackage{datetime}
\newdate{defensedate}{25}{10}{2017}

\usepackage{minted}

\usetheme[sectionpage=progressbar,subsectionpage=progressbar,numbering=fraction,
          progressbar=foot]{metropolis}

\usepackage{amssymb}

\title{Optimizing the integration of DPLL and $\mathcal{T}$-solvers}
% \subtitle{with a focus on lazy methods with interpolants}

\date{\displaydate{defensedate}}
\author{%
  Simon Bihel\hfill\url{simon.bihel@ens-rennes.fr} \\
}
\institute{%
  University of Rennes I \\
  \'Ecole Normale Sup\'erieure de Rennes
}

\begin{document}

\maketitle

\begin{frame}{Table of contents}
  \setbeamertemplate{section in toc}[sections numbered]
  \tableofcontents[hideallsubsections]
\end{frame}


\section{Basics of SMT($\mathcal{T}$) optimizations}

\begin{frame}{DPLL($\mathcal{T}$)}
  \begin{center}
    \includegraphics[width=0.95\textwidth]{LazySurvey_fig6}
  \end{center}
  \begin{description}
    \item[Offline] DPLL used as a SAT solver, re-invoked each time an assignment is found $\mathcal{T}$-unsatisfiable.
    \item[Online] DPLL used as an enumerator.
  \end{description}
\end{frame}

\begin{frame}{Preprocessing}
  Rewrite the input $\mathcal{T}$-formula into a $\mathcal{T}$-equivalent one.

  \begin{itemize}
    \item Eliminate $\mathcal{T}$-equivalent $\mathcal{T}$-atoms.
    \item Prevent ``obviously'' $\mathcal{T}$-inconsistent clauses.
  \end{itemize}

  \metroset{block=fill}
  \begin{exampleblock}{Example}
    Incompatible value assignment

    To prevent $\{x=0, x=1\}$, add $\neg (x=0) \lor \neg (x=1)$
  \end{exampleblock}
\end{frame}

\begin{frame}{Look-ahead}
  \begin{block}{Early pruning}
    Intermediate call to $\mathcal{T}$-solver on assignment construction.

    Avoids checking the $\mathcal{T}$-satisfiability of all total truth assignments which extend $\mu$.
  \end{block}
  \begin{block}{$\mathcal{T}$-propagation}
    $\mathcal{T}$-solver deducts literals on not-yet-assigned atoms.
  \end{block}
\end{frame}

\begin{frame}{Look-back}
  Generalizations of standard DPLL. Learns from conflict.
  \begin{description}
    \item[$\mathcal{T}$-backjumping] Backtracks to the highest point in the stack.
    \item[$\mathcal{T}$-learning] Prunes the highest number of future branches.
  \end{description}
\end{frame}

% \begin{frame}{Splitting on demand}
%   Delegate case-splits to the SAT solver.
% \end{frame}

\begin{frame}{Assignment simplification}
  \begin{description}
    \item[Clustering] ``Divide-and-conquer'' approach. Partition $\mathcal{T}$-atoms in clusters which do not interfere to each-other's $\mathcal{T}$-satisfiability.
    \item[Reduction of assignments] Removes literals unnecessary to satisfy $\varphi$.
    \item[Pure-literal filtering] Removes assignments with new polarity.
    \item[$\mathcal{T}$-deduced-literal filtering] Removes assignments propagated on $\mathcal{T}$-valid clauses because $\mu$ is $\mathcal{T}$-equisatisfiable.
  \end{description}
\end{frame}


\section{Lazy Abstraction with Interpolants}

\begin{frame}{Symbolic Model Checking}
  Check whether a model of a system meets a given specification.

  \includegraphics[width=0.9\textwidth]{interpolants_fig1.png}

  \begin{description}
    \item[Lazy abstraction] iteratively refine a model (First Order Logic)
  \end{description}
\end{frame}

\begin{frame}{Interpolants}
  \begin{description}
    \item[Interpolant] List of formulas derived from refutation of the path generated.
  \end{description}

  Given a pair of formulas $(A, B)$, such that $A \wedge B$ is inconsistent.
  An interpolant $\hat{A}$ has the following properties:
  \begin{itemize}
    \item $A$ implies $\hat{A}$, and
    \item $\hat{A} \wedge B$ is unsatisfiable.
  \end{itemize}

  \begin{block}{Craig interpolation lemma}
    An interpolant always exists for inconsistent formulas in FOL.
  \end{block}
\end{frame}

\begin{frame}{Program unwinding}
  \includegraphics[width=0.95\textwidth]{interpolants_fig2.png}

  For (a) it could be $\textsc{True}, L=0, \textsc{False}$.
\end{frame}

\begin{frame}{Other optimization}
  Program slicing, to remove parts of the program having no effect on feasibility of paths.
\end{frame}


% \section*{Conclusion}

% \begin{frame}{Conclusion}
%   Many tools for different situations.
% \end{frame}


\end{document}
